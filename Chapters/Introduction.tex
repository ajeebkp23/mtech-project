\begin{chapter}{Introduction}
    \thispagestyle{empty}
    In today's world of connected people using the Internet and social media, the chance of getting in touch with different language people is very normal. This project is an attempt to help people who are not familiar with Malayalam and are willing to take assistance from tools like Google Translate so they don't miss communication from Malayalam-speaking persons. So once they get a Manglish text they can utilize our system to convert Manglish to Malayalam script. Which then can be translated with other tools like Google Translator, DeepL translator, Bing Translator, Brave Translator, etc. 
    
    \begin{section}{Motivation}
        \begin{itemize}
  \item Language understanding is a major part of communication
  \item Manglish is the Malayalam sentences expressed in English
Alphabets
  \item Manglish is often used on Social Media platforms in forms like
comments, chats, and hashtags
  \item Various NLP tasks like classification, summarization, etc. might have a
better support for Malayalam than Manglish.
  \item Non-malayali people might be unable to understand the Manglish
language
  \item Fairly good amount of Malayalam to English translation is available.
  \item There is a need to convert Manglish to text to Malayalam.
\end{itemize}
    \end{section}

    \begin{section}{Problem Statement}
        \textbf{Aims to create an NMT Model that translates from
Manglish to English}

        \par This focus is to build a machine-learning system that is trained based on back-translated data. Since no body types both Manglish and Malayalam scripts sat a time, the chance for having a dataset is very less. So we need to get some synthetic data. It's shown in many research papers this back translation technique is effective. Since the model is built using Deep Learning techniques, We need to feed the proper data and the system automatically learns based on available data.
        \begin{subsection}{Objectives}
            \begin{itemize}
  \item Create a dataset based on OPUS Corpus
  \item Create or utilize word embeddings or other suitable word/subword representations
  \item Develop a model for translating from Manglish to Malayalam
  \item Prepare a simple interactive Demo for the model build
\end{itemize}
        \end{subsection}
    \end{section}

    \begin{section}{Major contribution of the Dissertation}
        This project explores various translation approaches used in the wild and provides a summarized view of research in the area. And find the best approach used by various researchers in the field and apply the best observations to solve our problem in the hand. In our observation, transformers based models are best in performance. So we build a transformers based system for doing the transliteration.
    \end{section}
    \begin{section}{Thesis Outline}
    \begin{enumerate} 
    \item \textbf{Chapter 1: Introduction} This chapter provides the groundwork for the research, explaining the motivation behind the study, identifying the research problem, outlining the objectives, and furnishing an overview of the methodology employed. \item \textbf{Chapter 2: Literature Survey} This chapter conducts a systematic examination of existing approaches in translation, thoroughly exploring existing methodologies while identifying gaps in the literature. 
    \item \textbf{Chapter 3: Proposed System} This chapter elaborates on the datasets employed in the research and explains the underlying principles of the various translation models. Furthermore, it advocates for repurposing neural machine translation for the Manglish to Malayalam transliteration. 
    \item \textbf{Chapter 4: Results and Discussion} This chapter evaluates the performance of the NMT method against alternative models. Additionally, major MT evaluation like BLEU score is explained. 
    \item \textbf{Chapter 5: Conclusion} This chapter summarizes the major contributions of the research, offering insights about the training, deployment, etc.
\end{enumerate}
    \end{section}
\end{chapter}