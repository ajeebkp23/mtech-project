\thispagestyle{plain}
\begin{center}
    \Large \bfseries  {ABSTRACT} \\[1cm]
\end{center}

\vspace{2\baselineskip}

\noindent
\par The goal of this project is to build a transliteration system that converts from Manglish script to Malayalam script. 

\par Transliteration is converting text from one script to another based on phonetic similarity, focusing on representing the sounds or characters of one writing system using the characters of another system. Unlike translation, which involves converting the meaning of words or phrases from one language to another, transliteration is primarily concerned with accurately preserving the pronunciation or visual representation of words or phrases in a different script without focusing on the meaning of the content

\par Manglish or Malayalam English, is a colloquial form of English spoken in the state of Kerala, India, which incorporates elements of Malayalam, the local language. Transliterating Manglish to Malayalam involves converting the English words or phrases spoken in the Manglish dialect to Malayalam script, based on the pronunciation and phonetic similarity between the two languages.

\par The development of a Manglish-to-Malayalam transliteration system offers benefits such as accurate pronunciation representation, enhanced accessibility, cultural preservation, improved communication, and efficient text processing for content written in Manglish.

\vspace{2\baselineskip}

\noindent
\textbf{Keywords}: Rule-based translation, Machine learning-based translation, neural machine translation, generative models backward translation, transliteration